\documentclass[finnish, a4paper, 12pt]{article}
\usepackage[finnish]{babel}
\usepackage{amsxtra,enumerate, amsmath, amsthm, amssymb}
\usepackage{bbm, xcolor}
\usepackage{verbatim} 
\usepackage[utf8]{inputenc} % tällä ääkkösiä

\usepackage{enumitem}
\usepackage{version}

%%%%%%%%%%%%%%%%%%% layout %%%%%%%%%%%%%%%%%%

\usepackage{geometry}
\geometry{top=1cm, bottom=2cm, left=2.5cm, right=2.5cm}

\parindent0em
\pagestyle{empty}

%\includeversion{version:withAnswers}
\excludeversion{version:withAnswers}

%%%%%%%%%%%%%%%%%%%%%%%%%%%%%%%%%%%%%%%%%

\begin{document}
	
	
%	\begin{flushright}
		PVM: \underline{\phantom{mm.mm.}}
		\hfill
		(L1T49G)	% tunniste
%	\end{flushright}
	
	\begin{center}
		{\large
			%{\bf Calculus 1} \\
			LASKUTAITOTESTI 1 (Calculus 1)}
	\end{center}
	
	Laskinta ei saa käyttää. Kaavakokoelmia ei saa käyttää.
	
	Kukin tehtävä arvostellaan pelkän vastauksen perusteella (oikein/väärin).
	Välivaiheita saa kirjoittaa näkyviin, kunhan vastaus on selvästi luettavissa.
	Suttupaperia saa käyttää. Kaikki käytetyt paperit palautetaan.
	
\vspace{12pt}
Nimi ja syntymäaika: \phantom{m} \hrulefill
\vspace{8pt}
	
	\begin{enumerate}[leftmargin=*]
		\setlength\itemsep{1em}
		
		\item %1  Desimaaliluku murtoluvuksi, helppo supistus
		Kirjoita murto- tai sekalukuna supistetussa muodossa. 
		
		\(
		0{,}88 = 
		\) % \frac{22}{25}
		
		\begin{version:withAnswers}
		\( \frac{22}{25} \)
		\end{version:withAnswers}

		\vspace{8pt}
		
		\item %2  Kymmenpotenssimuoto desimaaliluvuksi
		Kirjoita desimaalilukuna ilman kymmenpotenssimuotoa. 
		
		\(
		3{,}449168\cdot 10^{5} = 
		\) % 344916,8
		\begin{version:withAnswers}
		\( 344916,8 \)
		\end{version:withAnswers}	
		\vspace{8pt}
		
		\item %3  Murtolukujen summa tai erotus, lavennus ja supistus
		Laske. Kirjoita tulos murto- tai sekalukuna supistetussa muodossa.
		
		\(
		\displaystyle
		\frac{3}{4}-\frac{2}{3} = 
		\) % \frac{1}{12}
		\begin{version:withAnswers}
		\( \frac{1}{12} \)
		\end{version:withAnswers}	
		
		\vspace{8pt}
		
		\item %4  Murtolukujen kerto- tai jakolasku, supistus
		Laske. Kirjoita tulos murto- tai sekalukuna supistetussa muodossa.
		
		\(
		\displaystyle
		\frac{3}{4}\cdot\frac{7}{12} = 
		\) % \frac{3}{4}\cdot\frac{7}{12}
		\begin{version:withAnswers}
		\( \frac{7}{16} \)
		\end{version:withAnswers}
		
		\vspace{8pt}
		
		\item %5  Potenssin laskusäännöt
		Sievennä mahdollisimman yksinkertaiseen muotoon, kun \(a \not = 0\). 
		
		\(
		\displaystyle
		\frac{a^{7^2}\cdot a^2}{\left(a^6\right)^3} =
		\phantom{mmmmmmmmmmmmmmm}
		\) % a^{33}
		\begin{version:withAnswers}
		\(  a^{33} \)
		\end{version:withAnswers}
		
		\vspace{8pt}
		
		\item %6  Polynomilaskenta (välillä myös miinus sulkeiden edessä)
		Sievennä eli kirjoita lausekkeeksi, jossa ei esiinny sulkeita ja 
		samanmuotoiset termit on yhdistetty. 
		Kirjoita vastauksessa termit asteen mukaan alenevassa järjestyksessä. 
		% Tämä automaattista tarkistusta ennakoiden.
		
		\(
		\displaystyle
		x(7x^2 + x)(1 + 5x) = 
		\) % 35x^4 +12x^3 +x^2
		\begin{version:withAnswers}
		\( 35x^4 +12x^3 +x^2 \)
		\end{version:withAnswers}
		
		\vspace{8pt}
		
		\item %7 Rationaalilauseke, summa tai erotus (ei supistusta!)
		Sievennä %yhdeksi rationaalilausekkeeksi 
		muotoon, jossa esiintyy vain yksi jakoviiva ja sekä osoittaja
		että nimittäjä on sievennetty kuten edellisessä tehtävässä.
		
		\(
		\displaystyle
		\frac{2}{x+2}- \frac{2}{x + 3} =
		\) % \frac{2}{x^2+5x+6}
		\begin{version:withAnswers}
		\( \frac{2}{x^2+5x+6} \)
		\end{version:withAnswers}
		
		\vspace{8pt}
		
		\item %8 Yhtälön ratkaiseminen, 1. aste
		Ratkaise \(x\) yhtälöstä \(4x - (2x-4) = 3 - 5x\).
		
		\(
		x = 
		\)	%  \frac{-1}{7}
		\begin{version:withAnswers}
		 \( \frac{-1}{7} \)
		\end{version:withAnswers}
		\vspace{8pt}
		
		\item %9 Yhtälön ratkaiseminen, 2. aste
		Ratkaise \(x\) yhtälöstä \(4x^2+7x-2= 0\).
		
		\(
		x = 		
		\)	% \frac{1}{4} \text{  tai } x = -2
		
		\begin{version:withAnswers}
		\(\frac{1}{4} \text{  tai } x = -2\)
		\end{version:withAnswers}
		\vspace{8pt}
		
		\item %10 Yhtälön ratkaiseminen, monta kirjainta
		Ratkaise \(S\) yhtälöstä 
		\(
		\displaystyle \,
		kLm = \frac{nO+q}{rS} .
		\)	
		
		\(
		S = 
		\) %  \frac{nO+q}{rkLm}
		\begin{version:withAnswers}
		\( \frac{nO+q}{rkLm}\)
		\end{version:withAnswers}
		
	\end{enumerate}
	
	
\end{document}

