\documentclass[finnish, a4paper, 12pt]{article}
\usepackage[finnish]{babel}
\usepackage{amsxtra,enumerate, amsmath, amsthm, amssymb}
\usepackage{bbm, xcolor}
\usepackage{verbatim} 
\usepackage[utf8]{inputenc} % tällä ääkkösiä

\usepackage{enumitem}
\usepackage{version}

%%%%%%%%%%%%%%%%%%% layout %%%%%%%%%%%%%%%%%%

\usepackage{geometry}
\geometry{top=1cm, bottom=2cm, left=2.5cm, right=2.5cm}

\parindent0em
\pagestyle{empty}

%\includeversion{version:withAnswers}
\excludeversion{version:withAnswers}

%%%%%%%%%%%%%%%%%%%%%%%%%%%%%%%%%%%%%%%%%

\begin{document}
	
	
%	\begin{flushright}
		PVM: \underline{\phantom{mm.mm.}}
		\hfill
		(L1T28G)	% tunniste
%	\end{flushright}
	
	\begin{center}
		{\large
			%{\bf Calculus 1} \\
			LASKUTAITOTESTI 1 (Calculus 1)}
	\end{center}
	
	Laskinta ei saa käyttää. Kaavakokoelmia ei saa käyttää.
	
	Kukin tehtävä arvostellaan pelkän vastauksen perusteella (oikein/väärin).
	Välivaiheita saa kirjoittaa näkyviin, kunhan vastaus on selvästi luettavissa.
	Suttupaperia saa käyttää. Kaikki käytetyt paperit palautetaan.
	
\vspace{12pt}
Nimi ja syntymäaika: \phantom{m} \hrulefill
\vspace{8pt}
	
	\begin{enumerate}[leftmargin=*]
		\setlength\itemsep{1em}
		
		\item %1  Desimaaliluku murtoluvuksi, helppo supistus
		Kirjoita murto- tai sekalukuna supistetussa muodossa. 
		
		\(
		0{,}48 = 
		\) % \frac{12}{25}
		
		\begin{version:withAnswers}
		\( \frac{12}{25} \)
		\end{version:withAnswers}

		\vspace{8pt}
		
		\item %2  Kymmenpotenssimuoto desimaaliluvuksi
		Kirjoita desimaalilukuna ilman kymmenpotenssimuotoa. 
		
		\(
		2{,}248182\cdot 10^{3} = 
		\) % 2248,182
		\begin{version:withAnswers}
		\( 2248,182 \)
		\end{version:withAnswers}	
		\vspace{8pt}
		
		\item %3  Murtolukujen summa tai erotus, lavennus ja supistus
		Laske. Kirjoita tulos murto- tai sekalukuna supistetussa muodossa.
		
		\(
		\displaystyle
		\frac{7}{9}-\frac{2}{5} = 
		\) % \frac{17}{45}
		\begin{version:withAnswers}
		\( \frac{17}{45} \)
		\end{version:withAnswers}	
		
		\vspace{8pt}
		
		\item %4  Murtolukujen kerto- tai jakolasku, supistus
		Laske. Kirjoita tulos murto- tai sekalukuna supistetussa muodossa.
		
		\(
		\displaystyle
		\frac{2}{7}\cdot\frac{3}{8} = 
		\) % \frac{2}{7}\cdot\frac{3}{8}
		\begin{version:withAnswers}
		\( \frac{3}{28} \)
		\end{version:withAnswers}
		
		\vspace{8pt}
		
		\item %5  Potenssin laskusäännöt
		Sievennä mahdollisimman yksinkertaiseen muotoon, kun \(a \not = 0\). 
		
		\(
		\displaystyle
		\frac{\left(a^7\right)^3 \cdot a^{6^2}}{a^7} =
		\phantom{mmmmmmmmmmmmmmm}
		\) % a^{50}
		\begin{version:withAnswers}
		\(  a^{50} \)
		\end{version:withAnswers}
		
		\vspace{8pt}
		
		\item %6  Polynomilaskenta (välillä myös miinus sulkeiden edessä)
		Sievennä eli kirjoita lausekkeeksi, jossa ei esiinny sulkeita ja 
		samanmuotoiset termit on yhdistetty. 
		Kirjoita vastauksessa termit asteen mukaan alenevassa järjestyksessä. 
		% Tämä automaattista tarkistusta ennakoiden.
		
		\(
		\displaystyle
		-4x(2x^2 - x)(1 - x) = 
		\) % 8x^4 -12x^3 +4x^2
		\begin{version:withAnswers}
		\( 8x^4 -12x^3 +4x^2 \)
		\end{version:withAnswers}
		
		\vspace{8pt}
		
		\item %7 Rationaalilauseke, summa tai erotus (ei supistusta!)
		Sievennä %yhdeksi rationaalilausekkeeksi 
		muotoon, jossa esiintyy vain yksi jakoviiva ja sekä osoittaja
		että nimittäjä on sievennetty kuten edellisessä tehtävässä.
		
		\(
		\displaystyle
		\frac{2}{x+1}+ \frac{ 3}{x - 1} =
		\) % \frac{5x^2 + 1}{x^2  - 1}
		\begin{version:withAnswers}
		\( \frac{5x^2 + 1}{x^2  - 1} \)
		\end{version:withAnswers}
		
		\vspace{8pt}
		
		\item %8 Yhtälön ratkaiseminen, 1. aste
		Ratkaise \(x\) yhtälöstä \(9x - (5x-3) = 8 - 10x\).
		
		\(
		x = 
		\)	%  \frac{5}{14}
		\begin{version:withAnswers}
		 \( \frac{5}{14} \)
		\end{version:withAnswers}
		\vspace{8pt}
		
		\item %9 Yhtälön ratkaiseminen, 2. aste
		Ratkaise \(x\) yhtälöstä \(4 x^2  - 9 x + 5= 0\).
		
		\(
		x = 		
		\)	% \frac{5}{4} \text{  tai } x = 1
		
		\begin{version:withAnswers}
		\(\frac{5}{4} \text{  tai } x = 1\)
		\end{version:withAnswers}
		\vspace{8pt}
		
		\item %10 Yhtälön ratkaiseminen, monta kirjainta
		Ratkaise \(S\) yhtälöstä 
		\(
		\displaystyle \,
		zXy = \frac{vW+S}{uV} .
		\)	
		
		\(
		S = 
		\) %  zXyuV -vW
		\begin{version:withAnswers}
		\( zXyuV -vW\)
		\end{version:withAnswers}
		
	\end{enumerate}
	
	
\end{document}

