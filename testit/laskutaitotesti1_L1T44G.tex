\documentclass[finnish, a4paper, 12pt]{article}
\usepackage[finnish]{babel}
\usepackage{amsxtra,enumerate, amsmath, amsthm, amssymb}
\usepackage{bbm, xcolor}
\usepackage{verbatim} 
\usepackage[utf8]{inputenc} % tällä ääkkösiä

\usepackage{enumitem}
\usepackage{version}

%%%%%%%%%%%%%%%%%%% layout %%%%%%%%%%%%%%%%%%

\usepackage{geometry}
\geometry{top=1cm, bottom=2cm, left=2.5cm, right=2.5cm}

\parindent0em
\pagestyle{empty}

%\includeversion{version:withAnswers}
\excludeversion{version:withAnswers}

%%%%%%%%%%%%%%%%%%%%%%%%%%%%%%%%%%%%%%%%%

\begin{document}
	
	
%	\begin{flushright}
		PVM: \underline{\phantom{mm.mm.}}
		\hfill
		(L1T44G)	% tunniste
%	\end{flushright}
	
	\begin{center}
		{\large
			%{\bf Calculus 1} \\
			LASKUTAITOTESTI 1 (Calculus 1)}
	\end{center}
	
	Laskinta ei saa käyttää. Kaavakokoelmia ei saa käyttää.
	
	Kukin tehtävä arvostellaan pelkän vastauksen perusteella (oikein/väärin).
	Välivaiheita saa kirjoittaa näkyviin, kunhan vastaus on selvästi luettavissa.
	Suttupaperia saa käyttää. Kaikki käytetyt paperit palautetaan.
	
\vspace{12pt}
Nimi ja syntymäaika: \phantom{m} \hrulefill
\vspace{8pt}
	
	\begin{enumerate}[leftmargin=*]
		\setlength\itemsep{1em}
		
		\item %1  Desimaaliluku murtoluvuksi, helppo supistus
		Kirjoita murto- tai sekalukuna supistetussa muodossa. 
		
		\(
		0{,}28 = 
		\) % \frac{7}{25}
		
		\begin{version:withAnswers}
		\( \frac{7}{25} \)
		\end{version:withAnswers}

		\vspace{8pt}
		
		\item %2  Kymmenpotenssimuoto desimaaliluvuksi
		Kirjoita desimaalilukuna ilman kymmenpotenssimuotoa. 
		
		\(
		2{,}164199\cdot 10^{-4} = 
		\) % 0,0002164199
		\begin{version:withAnswers}
		\( 0,0002164199 \)
		\end{version:withAnswers}	
		\vspace{8pt}
		
		\item %3  Murtolukujen summa tai erotus, lavennus ja supistus
		Laske. Kirjoita tulos murto- tai sekalukuna supistetussa muodossa.
		
		\(
		\displaystyle
		\frac{7}{8}-\frac{5}{6} = 
		\) % \frac{1}{24}
		\begin{version:withAnswers}
		\( \frac{1}{24} \)
		\end{version:withAnswers}	
		
		\vspace{8pt}
		
		\item %4  Murtolukujen kerto- tai jakolasku, supistus
		Laske. Kirjoita tulos murto- tai sekalukuna supistetussa muodossa.
		
		\(
		\displaystyle
		\frac{3}{4}\cdot\frac{2}{5} = 
		\) % \frac{3}{4}\cdot\frac{2}{5}
		\begin{version:withAnswers}
		\( \frac{3}{10} \)
		\end{version:withAnswers}
		
		\vspace{8pt}
		
		\item %5  Potenssin laskusäännöt
		Sievennä mahdollisimman yksinkertaiseen muotoon, kun \(a \not = 0\). 
		
		\(
		\displaystyle
		\frac{\left(a^5\right)^2 \cdot a^{6^2}}{a^4} =
		\phantom{mmmmmmmmmmmmmmm}
		\) % a^{42}
		\begin{version:withAnswers}
		\(  a^{42} \)
		\end{version:withAnswers}
		
		\vspace{8pt}
		
		\item %6  Polynomilaskenta (välillä myös miinus sulkeiden edessä)
		Sievennä eli kirjoita lausekkeeksi, jossa ei esiinny sulkeita ja 
		samanmuotoiset termit on yhdistetty. 
		Kirjoita vastauksessa termit asteen mukaan alenevassa järjestyksessä. 
		% Tämä automaattista tarkistusta ennakoiden.
		
		\(
		\displaystyle
		-4x(3x^3 - 2x^2)(1 - 3x) = 
		\) % 36x^5 -36x^4 +8x^3
		\begin{version:withAnswers}
		\( 36x^5 -36x^4 +8x^3 \)
		\end{version:withAnswers}
		
		\vspace{8pt}
		
		\item %7 Rationaalilauseke, summa tai erotus (ei supistusta!)
		Sievennä %yhdeksi rationaalilausekkeeksi 
		muotoon, jossa esiintyy vain yksi jakoviiva ja sekä osoittaja
		että nimittäjä on sievennetty kuten edellisessä tehtävässä.
		
		\(
		\displaystyle
		\frac{1}{x+3}- \frac{1}{x - 1} =
		\) % \frac{- 4}{x^2  + 2x - 3}
		\begin{version:withAnswers}
		\( \frac{- 4}{x^2  + 2x - 3} \)
		\end{version:withAnswers}
		
		\vspace{8pt}
		
		\item %8 Yhtälön ratkaiseminen, 1. aste
		Ratkaise \(x\) yhtälöstä \(6x - (3x+4) = 5 - 7x\).
		
		\(
		x = 
		\)	%  \frac{9}{10}
		\begin{version:withAnswers}
		 \( \frac{9}{10} \)
		\end{version:withAnswers}
		\vspace{8pt}
		
		\item %9 Yhtälön ratkaiseminen, 2. aste
		Ratkaise \(x\) yhtälöstä \(5 x^2  + 9 x - 2= 0\).
		
		\(
		x = 		
		\)	% \frac{1}{5} \text{  tai } x = -2
		
		\begin{version:withAnswers}
		\(\frac{1}{5} \text{  tai } x = -2\)
		\end{version:withAnswers}
		\vspace{8pt}
		
		\item %10 Yhtälön ratkaiseminen, monta kirjainta
		Ratkaise \(S\) yhtälöstä 
		\(
		\displaystyle \,
		dEf = \frac{gS+i}{jK} .
		\)	
		
		\(
		S = 
		\) %  \frac{dEfjk-i}{g}
		\begin{version:withAnswers}
		\( \frac{dEfjk-i}{g}\)
		\end{version:withAnswers}
		
	\end{enumerate}
	
	
\end{document}

