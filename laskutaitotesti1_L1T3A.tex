\documentclass[finnish, a4paper, 12pt]{article}
\usepackage[finnish]{babel}
\usepackage{amsxtra,enumerate, amsmath, amsthm, amssymb}
\usepackage{bbm, xcolor}
\usepackage{verbatim} 
\usepackage[utf8]{inputenc} % tällä ääkkösiä

\usepackage{enumitem}

%%%%%%%%%%%%%%%%%%% layout %%%%%%%%%%%%%%%%%%

\usepackage{geometry}
\geometry{top=1cm, bottom=2cm, left=2.5cm, right=2.5cm}

\parindent0em
\pagestyle{empty}

%%%%%%%%%%%%%%%%%%%%%%%%%%%%%%%%%%%%%%%%%

\begin{document}
	
	
	\begin{flushright}
		(L1T3A)	% tunniste
	\end{flushright}
	
	\begin{center}
		{\large
			%{\bf Calculus 1} \\
			LASKUTAITOTESTI 1 (Calculus 1)}
	\end{center}
	
	Laskinta ei saa käyttää. Kaavakokoelmia ei saa käyttää.
	
	Kukin tehtävä arvostellaan pelkän vastauksen perusteella (oikein/väärin).
	Välivaiheita saa kirjoittaa näkyviin, kunhan vastaus on selvästi luettavissa.
	Suttupaperia saa käyttää. Kaikki käytetyt paperit palautetaan.
	
s\vspace{12pt}
Nimi ja syntymäaika: \phantom{m} \hrulefill
\vspace{8pt}
	
	\begin{enumerate}[leftmargin=*]
		\setlength\itemsep{1em}
		
		\item %1  Desimaaliluku murtoluvuksi, helppo supistus
		Kirjoita murto- tai sekalukuna supistetussa muodossa. 
		
		\(
		1.84 = 
		\) % \frac{46}{25}
		
		\vspace{8pt}
		
		\item %2  Kymmenpotenssimuoto desimaaliluvuksi
		Kirjoita desimaalilukuna ilman kymmenpotenssimuotoa. 
		
		\(
		3.827775\cdot 10^{-7} = 
		\) % 0,0000003827775
		
		\vspace{8pt}
		
		\item %3  Murtolukujen summa tai erotus, lavennus ja supistus
		Laske. Kirjoita tulos murto- tai sekalukuna supistetussa muodossa.
		
		\(
		\displaystyle
		\frac{2}{9}+\frac{5}{8} = 
		\) % \frac{61}{72}
		
		\vspace{8pt}
		
		\item %4  Murtolukujen kerto- tai jakolasku, supistus
		Laske. Kirjoita tulos murto- tai sekalukuna supistetussa muodossa.
		
		\(
		\displaystyle
		\frac{3}{4}\cdot\frac{2}{11} = 
		\) % \frac{3}{4}\cdot\frac{2}{11}
		
		\vspace{8pt}
		
		\item %5  Potenssin laskusäännöt
		Sievennä mahdollisimman yksinkertaiseen muotoon, kun \(a \not = 0\). 
		
		\(
		\displaystyle
		\frac{a^7 \cdot \left(a^3\right)^2}{a^{2^2}} =
		\phantom{mmmmmmmmmmmmmmm}
		\) % a^{9}
		
		\vspace{8pt}
		
		\item %6  Polynomilaskenta (välillä myös miinus sulkeiden edessä)
		Sievennä eli kirjoita lausekkeeksi, jossa ei esiinny sulkeita ja 
		samanmuotoiset termit on yhdistetty. 
		Kirjoita vastauksessa termit asteen mukaan alenevassa järjestyksessä. 
		% Tämä automaattista tarkistusta ennakoiden.
		
		\(
		\displaystyle
		;;T6;; = 
		\) % ;;T6ans;;
		
		\vspace{8pt}
		
		\item %7 Rationaalilauseke, summa tai erotus (ei supistusta!)
		Sievennä %yhdeksi rationaalilausekkeeksi 
		muotoon, jossa esiintyy vain yksi jakoviiva ja sekä osoittaja
		että nimittäjä on sievennetty kuten edellisessä tehtävässä.
		
		\(
		\displaystyle
		;;T7;; =
		\) % ;;T7ans;;
		
		\vspace{8pt}
		
		\item %8 Yhtälön ratkaiseminen, 1. aste
		Ratkaise \(x\) yhtälöstä \(;;T8;;\).
		
		\(
		x = 
		\)	% x = ;;T8ans;;
		
		\vspace{8pt}
		
		\item %9 Yhtälön ratkaiseminen, 2. aste
		Ratkaise \(x\) yhtälöstä \(4 x^2- 5 x + 1= 0\).
		
		\(
		x = 		
		\)	% x = 1/4 tai x = 1
		
		\vspace{8pt}
		
		\item %10 Yhtälön ratkaiseminen, monta kirjainta
		Ratkaise S yhtälöstä 
		\(
		\displaystyle \,
		;;T10;; .
		\)	
		
		\(
		S = 
		\) % ;;T10ans;;
		
	\end{enumerate}
	
	
\end{document}

\documentclass[finnish, a4paper, 12pt]{article}
\usepackage[finnish]{babel}
\usepackage{amsxtra,enumerate, amsmath, amsthm, amssymb}
\usepackage{bbm, xcolor}
\usepackage{verbatim} 
\usepackage[utf8]{inputenc} % tällä ääkkösiä

\usepackage{enumitem}

%%%%%%%%%%%%%%%%%%% layout %%%%%%%%%%%%%%%%%%

\usepackage{geometry}
\geometry{top=1cm, bottom=2cm, left=2.5cm, right=2.5cm}

\parindent0em
\pagestyle{empty}

%%%%%%%%%%%%%%%%%%%%%%%%%%%%%%%%%%%%%%%%%

\begin{document}
	
	
	\begin{flushright}
		(L1T3A)	% tunniste
	\end{flushright}
	
	\begin{center}
		{\large
			%{\bf Calculus 1} \\
			LASKUTAITOTESTI 1 (Calculus 1)}
	\end{center}
	
	Laskinta ei saa käyttää. Kaavakokoelmia ei saa käyttää.
	
	Kukin tehtävä arvostellaan pelkän vastauksen perusteella (oikein/väärin).
	Välivaiheita saa kirjoittaa näkyviin, kunhan vastaus on selvästi luettavissa.
	Suttupaperia saa käyttää. Kaikki käytetyt paperit palautetaan.
	
s\vspace{12pt}
Nimi ja syntymäaika: \phantom{m} \hrulefill
\vspace{8pt}
	
	\begin{enumerate}[leftmargin=*]
		\setlength\itemsep{1em}
		
		\item %1  Desimaaliluku murtoluvuksi, helppo supistus
		Kirjoita murto- tai sekalukuna supistetussa muodossa. 
		
		\(
		0.84 = 
		\) % \frac{21}{25}
		
		\vspace{8pt}
		
		\item %2  Kymmenpotenssimuoto desimaaliluvuksi
		Kirjoita desimaalilukuna ilman kymmenpotenssimuotoa. 
		
		\(
		1.121196\cdot 10^{-7} = 
		\) % 0,0000001121196
		
		\vspace{8pt}
		
		\item %3  Murtolukujen summa tai erotus, lavennus ja supistus
		Laske. Kirjoita tulos murto- tai sekalukuna supistetussa muodossa.
		
		\(
		\displaystyle
		\frac{5}{6}+\frac{3}{5} = 
		\) % \frac{43}{30}
		
		\vspace{8pt}
		
		\item %4  Murtolukujen kerto- tai jakolasku, supistus
		Laske. Kirjoita tulos murto- tai sekalukuna supistetussa muodossa.
		
		\(
		\displaystyle
		\frac{3}{8}\cdot\frac{7}{9} = 
		\) % \frac{3}{8}\cdot\frac{7}{9}
		
		\vspace{8pt}
		
		\item %5  Potenssin laskusäännöt
		Sievennä mahdollisimman yksinkertaiseen muotoon, kun \(a \not = 0\). 
		
		\(
		\displaystyle
		\frac{a^2 \cdot \left(a^5\right)^2}{a^{2^2}} =
		\phantom{mmmmmmmmmmmmmmm}
		\) % a^{8}
		
		\vspace{8pt}
		
		\item %6  Polynomilaskenta (välillä myös miinus sulkeiden edessä)
		Sievennä eli kirjoita lausekkeeksi, jossa ei esiinny sulkeita ja 
		samanmuotoiset termit on yhdistetty. 
		Kirjoita vastauksessa termit asteen mukaan alenevassa järjestyksessä. 
		% Tämä automaattista tarkistusta ennakoiden.
		
		\(
		\displaystyle
		;;T6;; = 
		\) % ;;T6ans;;
		
		\vspace{8pt}
		
		\item %7 Rationaalilauseke, summa tai erotus (ei supistusta!)
		Sievennä %yhdeksi rationaalilausekkeeksi 
		muotoon, jossa esiintyy vain yksi jakoviiva ja sekä osoittaja
		että nimittäjä on sievennetty kuten edellisessä tehtävässä.
		
		\(
		\displaystyle
		;;T7;; =
		\) % ;;T7ans;;
		
		\vspace{8pt}
		
		\item %8 Yhtälön ratkaiseminen, 1. aste
		Ratkaise \(x\) yhtälöstä \(;;T8;;\).
		
		\(
		x = 
		\)	% x = ;;T8ans;;
		
		\vspace{8pt}
		
		\item %9 Yhtälön ratkaiseminen, 2. aste
		Ratkaise \(x\) yhtälöstä \(3 x^2+ 4 x + 1= 0\).
		
		\(
		x = 		
		\)	% x = -1/3 tai x = -1
		
		\vspace{8pt}
		
		\item %10 Yhtälön ratkaiseminen, monta kirjainta
		Ratkaise S yhtälöstä 
		\(
		\displaystyle \,
		;;T10;; .
		\)	
		
		\(
		S = 
		\) % ;;T10ans;;
		
	\end{enumerate}
	
	
\end{document}

\documentclass[finnish, a4paper, 12pt]{article}
\usepackage[finnish]{babel}
\usepackage{amsxtra,enumerate, amsmath, amsthm, amssymb}
\usepackage{bbm, xcolor}
\usepackage{verbatim} 
\usepackage[utf8]{inputenc} % tällä ääkkösiä

\usepackage{enumitem}

%%%%%%%%%%%%%%%%%%% layout %%%%%%%%%%%%%%%%%%

\usepackage{geometry}
\geometry{top=1cm, bottom=2cm, left=2.5cm, right=2.5cm}

\parindent0em
\pagestyle{empty}

%%%%%%%%%%%%%%%%%%%%%%%%%%%%%%%%%%%%%%%%%

\begin{document}
	
	
	\begin{flushright}
		(L1T3A)	% tunniste
	\end{flushright}
	
	\begin{center}
		{\large
			%{\bf Calculus 1} \\
			LASKUTAITOTESTI 1 (Calculus 1)}
	\end{center}
	
	Laskinta ei saa käyttää. Kaavakokoelmia ei saa käyttää.
	
	Kukin tehtävä arvostellaan pelkän vastauksen perusteella (oikein/väärin).
	Välivaiheita saa kirjoittaa näkyviin, kunhan vastaus on selvästi luettavissa.
	Suttupaperia saa käyttää. Kaikki käytetyt paperit palautetaan.
	
s\vspace{12pt}
Nimi ja syntymäaika: \phantom{m} \hrulefill
\vspace{8pt}
	
	\begin{enumerate}[leftmargin=*]
		\setlength\itemsep{1em}
		
		\item %1  Desimaaliluku murtoluvuksi, helppo supistus
		Kirjoita murto- tai sekalukuna supistetussa muodossa. 
		
		\(
		0.72 = 
		\) % \frac{18}{25}
		
		\vspace{8pt}
		
		\item %2  Kymmenpotenssimuoto desimaaliluvuksi
		Kirjoita desimaalilukuna ilman kymmenpotenssimuotoa. 
		
		\(
		1.678817\cdot 10^{-7} = 
		\) % 0,0000001678817
		
		\vspace{8pt}
		
		\item %3  Murtolukujen summa tai erotus, lavennus ja supistus
		Laske. Kirjoita tulos murto- tai sekalukuna supistetussa muodossa.
		
		\(
		\displaystyle
		\frac{2}{3}+\frac{3}{4} = 
		\) % \frac{17}{12}
		
		\vspace{8pt}
		
		\item %4  Murtolukujen kerto- tai jakolasku, supistus
		Laske. Kirjoita tulos murto- tai sekalukuna supistetussa muodossa.
		
		\(
		\displaystyle
		\frac{7}{8}\cdot\frac{2}{3} = 
		\) % \frac{7}{8}\cdot\frac{2}{3}
		
		\vspace{8pt}
		
		\item %5  Potenssin laskusäännöt
		Sievennä mahdollisimman yksinkertaiseen muotoon, kun \(a \not = 0\). 
		
		\(
		\displaystyle
		\frac{a^2 \cdot \left(a^6\right)^3}{a^{4^2}} =
		\phantom{mmmmmmmmmmmmmmm}
		\) % a^{4}
		
		\vspace{8pt}
		
		\item %6  Polynomilaskenta (välillä myös miinus sulkeiden edessä)
		Sievennä eli kirjoita lausekkeeksi, jossa ei esiinny sulkeita ja 
		samanmuotoiset termit on yhdistetty. 
		Kirjoita vastauksessa termit asteen mukaan alenevassa järjestyksessä. 
		% Tämä automaattista tarkistusta ennakoiden.
		
		\(
		\displaystyle
		;;T6;; = 
		\) % ;;T6ans;;
		
		\vspace{8pt}
		
		\item %7 Rationaalilauseke, summa tai erotus (ei supistusta!)
		Sievennä %yhdeksi rationaalilausekkeeksi 
		muotoon, jossa esiintyy vain yksi jakoviiva ja sekä osoittaja
		että nimittäjä on sievennetty kuten edellisessä tehtävässä.
		
		\(
		\displaystyle
		;;T7;; =
		\) % ;;T7ans;;
		
		\vspace{8pt}
		
		\item %8 Yhtälön ratkaiseminen, 1. aste
		Ratkaise \(x\) yhtälöstä \(;;T8;;\).
		
		\(
		x = 
		\)	% x = ;;T8ans;;
		
		\vspace{8pt}
		
		\item %9 Yhtälön ratkaiseminen, 2. aste
		Ratkaise \(x\) yhtälöstä \(3 x^2  - 2 x - 8= 0\).
		
		\(
		x = 		
		\)	% x = -4/3 tai x = 2
		
		\vspace{8pt}
		
		\item %10 Yhtälön ratkaiseminen, monta kirjainta
		Ratkaise S yhtälöstä 
		\(
		\displaystyle \,
		;;T10;; .
		\)	
		
		\(
		S = 
		\) % ;;T10ans;;
		
	\end{enumerate}
	
	
\end{document}

\documentclass[finnish, a4paper, 12pt]{article}
\usepackage[finnish]{babel}
\usepackage{amsxtra,enumerate, amsmath, amsthm, amssymb}
\usepackage{bbm, xcolor}
\usepackage{verbatim} 
\usepackage[utf8]{inputenc} % tällä ääkkösiä

\usepackage{enumitem}

%%%%%%%%%%%%%%%%%%% layout %%%%%%%%%%%%%%%%%%

\usepackage{geometry}
\geometry{top=1cm, bottom=2cm, left=2.5cm, right=2.5cm}

\parindent0em
\pagestyle{empty}

%%%%%%%%%%%%%%%%%%%%%%%%%%%%%%%%%%%%%%%%%

\begin{document}
	
	
	\begin{flushright}
		(L1T3A)	% tunniste
	\end{flushright}
	
	\begin{center}
		{\large
			%{\bf Calculus 1} \\
			LASKUTAITOTESTI 1 (Calculus 1)}
	\end{center}
	
	Laskinta ei saa käyttää. Kaavakokoelmia ei saa käyttää.
	
	Kukin tehtävä arvostellaan pelkän vastauksen perusteella (oikein/väärin).
	Välivaiheita saa kirjoittaa näkyviin, kunhan vastaus on selvästi luettavissa.
	Suttupaperia saa käyttää. Kaikki käytetyt paperit palautetaan.
	
s\vspace{12pt}
Nimi ja syntymäaika: \phantom{m} \hrulefill
\vspace{8pt}
	
	\begin{enumerate}[leftmargin=*]
		\setlength\itemsep{1em}
		
		\item %1  Desimaaliluku murtoluvuksi, helppo supistus
		Kirjoita murto- tai sekalukuna supistetussa muodossa. 
		
		\(
		1.44 = 
		\) % \frac{36}{25}
		
		\vspace{8pt}
		
		\item %2  Kymmenpotenssimuoto desimaaliluvuksi
		Kirjoita desimaalilukuna ilman kymmenpotenssimuotoa. 
		
		\(
		1.230729\cdot 10^{-6} = 
		\) % 0,000001230729
		
		\vspace{8pt}
		
		\item %3  Murtolukujen summa tai erotus, lavennus ja supistus
		Laske. Kirjoita tulos murto- tai sekalukuna supistetussa muodossa.
		
		\(
		\displaystyle
		\frac{2}{9}+\frac{2}{7} = 
		\) % \frac{32}{63}
		
		\vspace{8pt}
		
		\item %4  Murtolukujen kerto- tai jakolasku, supistus
		Laske. Kirjoita tulos murto- tai sekalukuna supistetussa muodossa.
		
		\(
		\displaystyle
		\frac{3}{4}\cdot\frac{2}{5} = 
		\) % \frac{3}{4}\cdot\frac{2}{5}
		
		\vspace{8pt}
		
		\item %5  Potenssin laskusäännöt
		Sievennä mahdollisimman yksinkertaiseen muotoon, kun \(a \not = 0\). 
		
		\(
		\displaystyle
		\frac{a^{6^2}\cdot a^2}{\left(a^5\right)^2} =
		\phantom{mmmmmmmmmmmmmmm}
		\) % a^{28}
		
		\vspace{8pt}
		
		\item %6  Polynomilaskenta (välillä myös miinus sulkeiden edessä)
		Sievennä eli kirjoita lausekkeeksi, jossa ei esiinny sulkeita ja 
		samanmuotoiset termit on yhdistetty. 
		Kirjoita vastauksessa termit asteen mukaan alenevassa järjestyksessä. 
		% Tämä automaattista tarkistusta ennakoiden.
		
		\(
		\displaystyle
		;;T6;; = 
		\) % ;;T6ans;;
		
		\vspace{8pt}
		
		\item %7 Rationaalilauseke, summa tai erotus (ei supistusta!)
		Sievennä %yhdeksi rationaalilausekkeeksi 
		muotoon, jossa esiintyy vain yksi jakoviiva ja sekä osoittaja
		että nimittäjä on sievennetty kuten edellisessä tehtävässä.
		
		\(
		\displaystyle
		;;T7;; =
		\) % ;;T7ans;;
		
		\vspace{8pt}
		
		\item %8 Yhtälön ratkaiseminen, 1. aste
		Ratkaise \(x\) yhtälöstä \(;;T8;;\).
		
		\(
		x = 
		\)	% x = ;;T8ans;;
		
		\vspace{8pt}
		
		\item %9 Yhtälön ratkaiseminen, 2. aste
		Ratkaise \(x\) yhtälöstä \(5 x^2  - 9 x - 2= 0\).
		
		\(
		x = 		
		\)	% x = -1/5 tai x = 2
		
		\vspace{8pt}
		
		\item %10 Yhtälön ratkaiseminen, monta kirjainta
		Ratkaise S yhtälöstä 
		\(
		\displaystyle \,
		;;T10;; .
		\)	
		
		\(
		S = 
		\) % ;;T10ans;;
		
	\end{enumerate}
	
	
\end{document}

\documentclass[finnish, a4paper, 12pt]{article}
\usepackage[finnish]{babel}
\usepackage{amsxtra,enumerate, amsmath, amsthm, amssymb}
\usepackage{bbm, xcolor}
\usepackage{verbatim} 
\usepackage[utf8]{inputenc} % tällä ääkkösiä

\usepackage{enumitem}

%%%%%%%%%%%%%%%%%%% layout %%%%%%%%%%%%%%%%%%

\usepackage{geometry}
\geometry{top=1cm, bottom=2cm, left=2.5cm, right=2.5cm}

\parindent0em
\pagestyle{empty}

%%%%%%%%%%%%%%%%%%%%%%%%%%%%%%%%%%%%%%%%%

\begin{document}
	
	
	\begin{flushright}
		(L1T3A)	% tunniste
	\end{flushright}
	
	\begin{center}
		{\large
			%{\bf Calculus 1} \\
			LASKUTAITOTESTI 1 (Calculus 1)}
	\end{center}
	
	Laskinta ei saa käyttää. Kaavakokoelmia ei saa käyttää.
	
	Kukin tehtävä arvostellaan pelkän vastauksen perusteella (oikein/väärin).
	Välivaiheita saa kirjoittaa näkyviin, kunhan vastaus on selvästi luettavissa.
	Suttupaperia saa käyttää. Kaikki käytetyt paperit palautetaan.
	
s\vspace{12pt}
Nimi ja syntymäaika: \phantom{m} \hrulefill
\vspace{8pt}
	
	\begin{enumerate}[leftmargin=*]
		\setlength\itemsep{1em}
		
		\item %1  Desimaaliluku murtoluvuksi, helppo supistus
		Kirjoita murto- tai sekalukuna supistetussa muodossa. 
		
		\(
		0.88 = 
		\) % \frac{22}{25}
		
		\vspace{8pt}
		
		\item %2  Kymmenpotenssimuoto desimaaliluvuksi
		Kirjoita desimaalilukuna ilman kymmenpotenssimuotoa. 
		
		\(
		1.764657\cdot 10^{-8} = 
		\) % 0,00000001764657
		
		\vspace{8pt}
		
		\item %3  Murtolukujen summa tai erotus, lavennus ja supistus
		Laske. Kirjoita tulos murto- tai sekalukuna supistetussa muodossa.
		
		\(
		\displaystyle
		\frac{5}{6}-\frac{3}{5} = 
		\) % \frac{7}{30}
		
		\vspace{8pt}
		
		\item %4  Murtolukujen kerto- tai jakolasku, supistus
		Laske. Kirjoita tulos murto- tai sekalukuna supistetussa muodossa.
		
		\(
		\displaystyle
		\frac{5}{6}\cdot\frac{3}{4} = 
		\) % \frac{5}{6}\cdot\frac{3}{4}
		
		\vspace{8pt}
		
		\item %5  Potenssin laskusäännöt
		Sievennä mahdollisimman yksinkertaiseen muotoon, kun \(a \not = 0\). 
		
		\(
		\displaystyle
		\frac{a^{7^2}\cdot a^4}{\left(a^6\right)^3} =
		\phantom{mmmmmmmmmmmmmmm}
		\) % a^{35}
		
		\vspace{8pt}
		
		\item %6  Polynomilaskenta (välillä myös miinus sulkeiden edessä)
		Sievennä eli kirjoita lausekkeeksi, jossa ei esiinny sulkeita ja 
		samanmuotoiset termit on yhdistetty. 
		Kirjoita vastauksessa termit asteen mukaan alenevassa järjestyksessä. 
		% Tämä automaattista tarkistusta ennakoiden.
		
		\(
		\displaystyle
		;;T6;; = 
		\) % ;;T6ans;;
		
		\vspace{8pt}
		
		\item %7 Rationaalilauseke, summa tai erotus (ei supistusta!)
		Sievennä %yhdeksi rationaalilausekkeeksi 
		muotoon, jossa esiintyy vain yksi jakoviiva ja sekä osoittaja
		että nimittäjä on sievennetty kuten edellisessä tehtävässä.
		
		\(
		\displaystyle
		;;T7;; =
		\) % ;;T7ans;;
		
		\vspace{8pt}
		
		\item %8 Yhtälön ratkaiseminen, 1. aste
		Ratkaise \(x\) yhtälöstä \(;;T8;;\).
		
		\(
		x = 
		\)	% x = ;;T8ans;;
		
		\vspace{8pt}
		
		\item %9 Yhtälön ratkaiseminen, 2. aste
		Ratkaise \(x\) yhtälöstä \(6 x^2  - 13 x + 2= 0\).
		
		\(
		x = 		
		\)	% x = 1/6 tai x = 2
		
		\vspace{8pt}
		
		\item %10 Yhtälön ratkaiseminen, monta kirjainta
		Ratkaise S yhtälöstä 
		\(
		\displaystyle \,
		;;T10;; .
		\)	
		
		\(
		S = 
		\) % ;;T10ans;;
		
	\end{enumerate}
	
	
\end{document}

\documentclass[finnish, a4paper, 12pt]{article}
\usepackage[finnish]{babel}
\usepackage{amsxtra,enumerate, amsmath, amsthm, amssymb}
\usepackage{bbm, xcolor}
\usepackage{verbatim} 
\usepackage[utf8]{inputenc} % tällä ääkkösiä

\usepackage{enumitem}

%%%%%%%%%%%%%%%%%%% layout %%%%%%%%%%%%%%%%%%

\usepackage{geometry}
\geometry{top=1cm, bottom=2cm, left=2.5cm, right=2.5cm}

\parindent0em
\pagestyle{empty}

%%%%%%%%%%%%%%%%%%%%%%%%%%%%%%%%%%%%%%%%%

\begin{document}
	
	
	\begin{flushright}
		(L1T3A)	% tunniste
	\end{flushright}
	
	\begin{center}
		{\large
			%{\bf Calculus 1} \\
			LASKUTAITOTESTI 1 (Calculus 1)}
	\end{center}
	
	Laskinta ei saa käyttää. Kaavakokoelmia ei saa käyttää.
	
	Kukin tehtävä arvostellaan pelkän vastauksen perusteella (oikein/väärin).
	Välivaiheita saa kirjoittaa näkyviin, kunhan vastaus on selvästi luettavissa.
	Suttupaperia saa käyttää. Kaikki käytetyt paperit palautetaan.
	
s\vspace{12pt}
Nimi ja syntymäaika: \phantom{m} \hrulefill
\vspace{8pt}
	
	\begin{enumerate}[leftmargin=*]
		\setlength\itemsep{1em}
		
		\item %1  Desimaaliluku murtoluvuksi, helppo supistus
		Kirjoita murto- tai sekalukuna supistetussa muodossa. 
		
		\(
		1.16 = 
		\) % \frac{23}{20}
		
		\vspace{8pt}
		
		\item %2  Kymmenpotenssimuoto desimaaliluvuksi
		Kirjoita desimaalilukuna ilman kymmenpotenssimuotoa. 
		
		\(
		3.507985\cdot 10^{-5} = 
		\) % 0,00003507985
		
		\vspace{8pt}
		
		\item %3  Murtolukujen summa tai erotus, lavennus ja supistus
		Laske. Kirjoita tulos murto- tai sekalukuna supistetussa muodossa.
		
		\(
		\displaystyle
		\frac{8}{9}-\frac{5}{7} = 
		\) % \frac{11}{63}
		
		\vspace{8pt}
		
		\item %4  Murtolukujen kerto- tai jakolasku, supistus
		Laske. Kirjoita tulos murto- tai sekalukuna supistetussa muodossa.
		
		\(
		\displaystyle
		\frac{9}{10}\cdot\frac{2}{7} = 
		\) % \frac{9}{10}\cdot\frac{2}{7}
		
		\vspace{8pt}
		
		\item %5  Potenssin laskusäännöt
		Sievennä mahdollisimman yksinkertaiseen muotoon, kun \(a \not = 0\). 
		
		\(
		\displaystyle
		\frac{a^7 \cdot \left(a^3\right)^2}{a^{2^2}} =
		\phantom{mmmmmmmmmmmmmmm}
		\) % a^{9}
		
		\vspace{8pt}
		
		\item %6  Polynomilaskenta (välillä myös miinus sulkeiden edessä)
		Sievennä eli kirjoita lausekkeeksi, jossa ei esiinny sulkeita ja 
		samanmuotoiset termit on yhdistetty. 
		Kirjoita vastauksessa termit asteen mukaan alenevassa järjestyksessä. 
		% Tämä automaattista tarkistusta ennakoiden.
		
		\(
		\displaystyle
		;;T6;; = 
		\) % ;;T6ans;;
		
		\vspace{8pt}
		
		\item %7 Rationaalilauseke, summa tai erotus (ei supistusta!)
		Sievennä %yhdeksi rationaalilausekkeeksi 
		muotoon, jossa esiintyy vain yksi jakoviiva ja sekä osoittaja
		että nimittäjä on sievennetty kuten edellisessä tehtävässä.
		
		\(
		\displaystyle
		;;T7;; =
		\) % ;;T7ans;;
		
		\vspace{8pt}
		
		\item %8 Yhtälön ratkaiseminen, 1. aste
		Ratkaise \(x\) yhtälöstä \(;;T8;;\).
		
		\(
		x = 
		\)	% x = ;;T8ans;;
		
		\vspace{8pt}
		
		\item %9 Yhtälön ratkaiseminen, 2. aste
		Ratkaise \(x\) yhtälöstä \(5 x^2  + 6 x + 1= 0\).
		
		\(
		x = 		
		\)	% x = -1/5 tai x = -1
		
		\vspace{8pt}
		
		\item %10 Yhtälön ratkaiseminen, monta kirjainta
		Ratkaise S yhtälöstä 
		\(
		\displaystyle \,
		;;T10;; .
		\)	
		
		\(
		S = 
		\) % ;;T10ans;;
		
	\end{enumerate}
	
	
\end{document}

\documentclass[finnish, a4paper, 12pt]{article}
\usepackage[finnish]{babel}
\usepackage{amsxtra,enumerate, amsmath, amsthm, amssymb}
\usepackage{bbm, xcolor}
\usepackage{verbatim} 
\usepackage[utf8]{inputenc} % tällä ääkkösiä

\usepackage{enumitem}

%%%%%%%%%%%%%%%%%%% layout %%%%%%%%%%%%%%%%%%

\usepackage{geometry}
\geometry{top=1cm, bottom=2cm, left=2.5cm, right=2.5cm}

\parindent0em
\pagestyle{empty}

%%%%%%%%%%%%%%%%%%%%%%%%%%%%%%%%%%%%%%%%%

\begin{document}
	
	
	\begin{flushright}
		(L1T3A)	% tunniste
	\end{flushright}
	
	\begin{center}
		{\large
			%{\bf Calculus 1} \\
			LASKUTAITOTESTI 1 (Calculus 1)}
	\end{center}
	
	Laskinta ei saa käyttää. Kaavakokoelmia ei saa käyttää.
	
	Kukin tehtävä arvostellaan pelkän vastauksen perusteella (oikein/väärin).
	Välivaiheita saa kirjoittaa näkyviin, kunhan vastaus on selvästi luettavissa.
	Suttupaperia saa käyttää. Kaikki käytetyt paperit palautetaan.
	
s\vspace{12pt}
Nimi ja syntymäaika: \phantom{m} \hrulefill
\vspace{8pt}
	
	\begin{enumerate}[leftmargin=*]
		\setlength\itemsep{1em}
		
		\item %1  Desimaaliluku murtoluvuksi, helppo supistus
		Kirjoita murto- tai sekalukuna supistetussa muodossa. 
		
		\(
		1.08 = 
		\) % \frac{27}{25}
		
		\vspace{8pt}
		
		\item %2  Kymmenpotenssimuoto desimaaliluvuksi
		Kirjoita desimaalilukuna ilman kymmenpotenssimuotoa. 
		
		\(
		3.467531\cdot 10^{-5} = 
		\) % 0,00003467531
		
		\vspace{8pt}
		
		\item %3  Murtolukujen summa tai erotus, lavennus ja supistus
		Laske. Kirjoita tulos murto- tai sekalukuna supistetussa muodossa.
		
		\(
		\displaystyle
		\frac{7}{9}-\frac{3}{8} = 
		\) % \frac{29}{72}
		
		\vspace{8pt}
		
		\item %4  Murtolukujen kerto- tai jakolasku, supistus
		Laske. Kirjoita tulos murto- tai sekalukuna supistetussa muodossa.
		
		\(
		\displaystyle
		\frac{2}{7}\cdot\frac{3}{10} = 
		\) % \frac{2}{7}\cdot\frac{3}{10}
		
		\vspace{8pt}
		
		\item %5  Potenssin laskusäännöt
		Sievennä mahdollisimman yksinkertaiseen muotoon, kun \(a \not = 0\). 
		
		\(
		\displaystyle
		\frac{\left(a^3\right)^2 \cdot a^{2^2}}{a^2} =
		\phantom{mmmmmmmmmmmmmmm}
		\) % a^{8}
		
		\vspace{8pt}
		
		\item %6  Polynomilaskenta (välillä myös miinus sulkeiden edessä)
		Sievennä eli kirjoita lausekkeeksi, jossa ei esiinny sulkeita ja 
		samanmuotoiset termit on yhdistetty. 
		Kirjoita vastauksessa termit asteen mukaan alenevassa järjestyksessä. 
		% Tämä automaattista tarkistusta ennakoiden.
		
		\(
		\displaystyle
		;;T6;; = 
		\) % ;;T6ans;;
		
		\vspace{8pt}
		
		\item %7 Rationaalilauseke, summa tai erotus (ei supistusta!)
		Sievennä %yhdeksi rationaalilausekkeeksi 
		muotoon, jossa esiintyy vain yksi jakoviiva ja sekä osoittaja
		että nimittäjä on sievennetty kuten edellisessä tehtävässä.
		
		\(
		\displaystyle
		;;T7;; =
		\) % ;;T7ans;;
		
		\vspace{8pt}
		
		\item %8 Yhtälön ratkaiseminen, 1. aste
		Ratkaise \(x\) yhtälöstä \(;;T8;;\).
		
		\(
		x = 
		\)	% x = ;;T8ans;;
		
		\vspace{8pt}
		
		\item %9 Yhtälön ratkaiseminen, 2. aste
		Ratkaise \(x\) yhtälöstä \(4 x^2  + 3 x - 1= 0\).
		
		\(
		x = 		
		\)	% x = 1/4 tai x = -1
		
		\vspace{8pt}
		
		\item %10 Yhtälön ratkaiseminen, monta kirjainta
		Ratkaise S yhtälöstä 
		\(
		\displaystyle \,
		;;T10;; .
		\)	
		
		\(
		S = 
		\) % ;;T10ans;;
		
	\end{enumerate}
	
	
\end{document}

\documentclass[finnish, a4paper, 12pt]{article}
\usepackage[finnish]{babel}
\usepackage{amsxtra,enumerate, amsmath, amsthm, amssymb}
\usepackage{bbm, xcolor}
\usepackage{verbatim} 
\usepackage[utf8]{inputenc} % tällä ääkkösiä

\usepackage{enumitem}

%%%%%%%%%%%%%%%%%%% layout %%%%%%%%%%%%%%%%%%

\usepackage{geometry}
\geometry{top=1cm, bottom=2cm, left=2.5cm, right=2.5cm}

\parindent0em
\pagestyle{empty}

%%%%%%%%%%%%%%%%%%%%%%%%%%%%%%%%%%%%%%%%%

\begin{document}
	
	
	\begin{flushright}
		(L1T3A)	% tunniste
	\end{flushright}
	
	\begin{center}
		{\large
			%{\bf Calculus 1} \\
			LASKUTAITOTESTI 1 (Calculus 1)}
	\end{center}
	
	Laskinta ei saa käyttää. Kaavakokoelmia ei saa käyttää.
	
	Kukin tehtävä arvostellaan pelkän vastauksen perusteella (oikein/väärin).
	Välivaiheita saa kirjoittaa näkyviin, kunhan vastaus on selvästi luettavissa.
	Suttupaperia saa käyttää. Kaikki käytetyt paperit palautetaan.
	
s\vspace{12pt}
Nimi ja syntymäaika: \phantom{m} \hrulefill
\vspace{8pt}
	
	\begin{enumerate}[leftmargin=*]
		\setlength\itemsep{1em}
		
		\item %1  Desimaaliluku murtoluvuksi, helppo supistus
		Kirjoita murto- tai sekalukuna supistetussa muodossa. 
		
		\(
		1.32 = 
		\) % \frac{33}{25}
		
		\vspace{8pt}
		
		\item %2  Kymmenpotenssimuoto desimaaliluvuksi
		Kirjoita desimaalilukuna ilman kymmenpotenssimuotoa. 
		
		\(
		2.025606\cdot 10^{-8} = 
		\) % 0,00000002025606
		
		\vspace{8pt}
		
		\item %3  Murtolukujen summa tai erotus, lavennus ja supistus
		Laske. Kirjoita tulos murto- tai sekalukuna supistetussa muodossa.
		
		\(
		\displaystyle
		\frac{3}{4}+\frac{2}{5} = 
		\) % \frac{23}{20}
		
		\vspace{8pt}
		
		\item %4  Murtolukujen kerto- tai jakolasku, supistus
		Laske. Kirjoita tulos murto- tai sekalukuna supistetussa muodossa.
		
		\(
		\displaystyle
		\frac{3}{4}\cdot\frac{6}{11} = 
		\) % \frac{3}{4}\cdot\frac{6}{11}
		
		\vspace{8pt}
		
		\item %5  Potenssin laskusäännöt
		Sievennä mahdollisimman yksinkertaiseen muotoon, kun \(a \not = 0\). 
		
		\(
		\displaystyle
		\frac{a^{7^2}\cdot a^6}{\left(a^5\right)^3} =
		\phantom{mmmmmmmmmmmmmmm}
		\) % a^{40}
		
		\vspace{8pt}
		
		\item %6  Polynomilaskenta (välillä myös miinus sulkeiden edessä)
		Sievennä eli kirjoita lausekkeeksi, jossa ei esiinny sulkeita ja 
		samanmuotoiset termit on yhdistetty. 
		Kirjoita vastauksessa termit asteen mukaan alenevassa järjestyksessä. 
		% Tämä automaattista tarkistusta ennakoiden.
		
		\(
		\displaystyle
		;;T6;; = 
		\) % ;;T6ans;;
		
		\vspace{8pt}
		
		\item %7 Rationaalilauseke, summa tai erotus (ei supistusta!)
		Sievennä %yhdeksi rationaalilausekkeeksi 
		muotoon, jossa esiintyy vain yksi jakoviiva ja sekä osoittaja
		että nimittäjä on sievennetty kuten edellisessä tehtävässä.
		
		\(
		\displaystyle
		;;T7;; =
		\) % ;;T7ans;;
		
		\vspace{8pt}
		
		\item %8 Yhtälön ratkaiseminen, 1. aste
		Ratkaise \(x\) yhtälöstä \(;;T8;;\).
		
		\(
		x = 
		\)	% x = ;;T8ans;;
		
		\vspace{8pt}
		
		\item %9 Yhtälön ratkaiseminen, 2. aste
		Ratkaise \(x\) yhtälöstä \(3 x^2  + x - 2= 0\).
		
		\(
		x = 		
		\)	% x = 2/3 tai x = -1
		
		\vspace{8pt}
		
		\item %10 Yhtälön ratkaiseminen, monta kirjainta
		Ratkaise S yhtälöstä 
		\(
		\displaystyle \,
		;;T10;; .
		\)	
		
		\(
		S = 
		\) % ;;T10ans;;
		
	\end{enumerate}
	
	
\end{document}

