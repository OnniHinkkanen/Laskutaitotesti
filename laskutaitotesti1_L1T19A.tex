\documentclass[finnish, a4paper, 12pt]{article}
\usepackage[finnish]{babel}
\usepackage{amsxtra,enumerate, amsmath, amsthm, amssymb}
\usepackage{bbm, xcolor}
\usepackage{verbatim} 
\usepackage[utf8]{inputenc} % tällä ääkkösiä

\usepackage{enumitem}

%%%%%%%%%%%%%%%%%%% layout %%%%%%%%%%%%%%%%%%

\usepackage{geometry}
\geometry{top=1cm, bottom=2cm, left=2.5cm, right=2.5cm}

\parindent0em
\pagestyle{empty}

%%%%%%%%%%%%%%%%%%%%%%%%%%%%%%%%%%%%%%%%%

\begin{document}
	
	
	\begin{flushright}
		(L1T19A)	% tunniste
	\end{flushright}
	
	\begin{center}
		{\large
			%{\bf Calculus 1} \\
			LASKUTAITOTESTI 1 (Calculus 1)}
	\end{center}
	
	Laskinta ei saa käyttää. Kaavakokoelmia ei saa käyttää.
	
	Kukin tehtävä arvostellaan pelkän vastauksen perusteella (oikein/väärin).
	Välivaiheita saa kirjoittaa näkyviin, kunhan vastaus on selvästi luettavissa.
	Suttupaperia saa käyttää. Kaikki käytetyt paperit palautetaan.
	
s\vspace{12pt}
Nimi ja syntymäaika: \phantom{m} \hrulefill
\vspace{8pt}
	
	\begin{enumerate}[leftmargin=*]
		\setlength\itemsep{1em}
		
		\item %1  Desimaaliluku murtoluvuksi, helppo supistus
		Kirjoita murto- tai sekalukuna supistetussa muodossa. 
		
		\(
		0.36 = 
		\) % \frac{9}{25}
		
		\vspace{8pt}
		
		\item %2  Kymmenpotenssimuoto desimaaliluvuksi
		Kirjoita desimaalilukuna ilman kymmenpotenssimuotoa. 
		
		\(
		2.310070\cdot 10^{-6} = 
		\) % 0,000002310070
		
		\vspace{8pt}
		
		\item %3  Murtolukujen summa tai erotus, lavennus ja supistus
		Laske. Kirjoita tulos murto- tai sekalukuna supistetussa muodossa.
		
		\(
		\displaystyle
		\frac{3}{4}-\frac{4}{5} = 
		\) % \frac{1}{-20}
		
		\vspace{8pt}
		
		\item %4  Murtolukujen kerto- tai jakolasku, supistus
		Laske. Kirjoita tulos murto- tai sekalukuna supistetussa muodossa.
		
		\(
		\displaystyle
		\frac{7}{10}\cdot\frac{2}{5} = 
		\) % \frac{7}{10}\cdot\frac{2}{5}
		
		\vspace{8pt}
		
		\item %5  Potenssin laskusäännöt
		Sievennä mahdollisimman yksinkertaiseen muotoon, kun \(a \not = 0\). 
		
		\(
		\displaystyle
		\frac{\left(a^2\right)^3 \cdot a^{6^2}}{a^6} =
		\phantom{mmmmmmmmmmmmmmm}
		\) % a^{36}
		
		\vspace{8pt}
		
		\item %6  Polynomilaskenta (välillä myös miinus sulkeiden edessä)
		Sievennä eli kirjoita lausekkeeksi, jossa ei esiinny sulkeita ja 
		samanmuotoiset termit on yhdistetty. 
		Kirjoita vastauksessa termit asteen mukaan alenevassa järjestyksessä. 
		% Tämä automaattista tarkistusta ennakoiden.
		
		\(
		\displaystyle
		x(7x^2 + x) + (1 + 5x) = 
		\) % 7x^3 +x^2 +5x +1
		
		\vspace{8pt}
		
		\item %7 Rationaalilauseke, summa tai erotus (ei supistusta!)
		Sievennä %yhdeksi rationaalilausekkeeksi 
		muotoon, jossa esiintyy vain yksi jakoviiva ja sekä osoittaja
		että nimittäjä on sievennetty kuten edellisessä tehtävässä.
		
		\(
		\displaystyle
		\frac{2}{x-1}- \frac{1}{x + 1} =
		\) % \frac{x^2 + 3}{x^2  - 1}
		
		\vspace{8pt}
		
		\item %8 Yhtälön ratkaiseminen, 1. aste
		Ratkaise \(x\) yhtälöstä \(2x + 3(6 - x) = 12\).
		
		\(
		x = 
		\)	% x = x = 6
		
		\vspace{8pt}
		
		\item %9 Yhtälön ratkaiseminen, 2. aste
		Ratkaise \(x\) yhtälöstä \(3 x^2  - x - 10= 0\).
		
		\(
		x = 		
		\)	% x = -5/3 tai x = 2
		
		\vspace{8pt}
		
		\item %10 Yhtälön ratkaiseminen, monta kirjainta
		Ratkaise S yhtälöstä 
		\(
		\displaystyle \,
		kLm = \frac{nO+q}{rS} .
		\)	
		
		\(
		S = 
		\) % S = \frac{nO+q}{rkLm}
		
	\end{enumerate}
	
	
\end{document}

